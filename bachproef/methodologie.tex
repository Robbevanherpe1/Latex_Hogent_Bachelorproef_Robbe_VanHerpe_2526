%%=============================================================================
%% Methodologie
%%=============================================================================

\chapter{\IfLanguageName{dutch}{Methodologie}{Methodology}}%
\label{ch:methodologie}

%% TODO: In dit hoofstuk geef je een korte toelichting over hoe je te werk bent
%% gegaan. Verdeel je onderzoek in grote fasen, en licht in elke fase toe wat
%% de doelstelling was, welke deliverables daar uit gekomen zijn, en welke
%% onderzoeksmethoden je daarbij toegepast hebt. Verantwoord waarom je
%% op deze manier te werk gegaan bent.
%% 
%% Voorbeelden van zulke fasen zijn: literatuurstudie, opstellen van een
%% requirements-analyse, opstellen long-list (bij vergelijkende studie),
%% selectie van geschikte tools (bij vergelijkende studie, "short-list"),
%% opzetten testopstelling/PoC, uitvoeren testen en verzamelen
%% van resultaten, analyse van resultaten, ...
%%
%% !!!!! LET OP !!!!!
%%
%% Het is uitdrukkelijk NIET de bedoeling dat je het grootste deel van de corpus
%% van je bachelorproef in dit hoofstuk verwerkt! Dit hoofdstuk is eerder een
%% kort overzicht van je plan van aanpak.
%%
%% Maak voor elke fase (behalve het literatuuronderzoek) een NIEUW HOOFDSTUK aan
%% en geef het een gepaste titel.



Het uitwerken van een competentieframework en een leersite gebeurt in verschillende stappen.
In dit hoofdstuk wordt kort toegelicht welke fasen en methoden in deze bachelorproef worden gevolgd
om tot een onderbouwd en bruikbaar resultaat te komen.


\section{Opbouw van de methodologie}%
\label{sec:opbouw-van-de-methodologie}


Deze bachelorproef is opgebouwd uit vier fasen:

fase 1: analyse voor het bepalen van de competenties binnen het framework en de requirements van de proof of concept
fase 2: opstellen van het competentieframework
fase 3: ontwikkelen van de proof of concept
fase 4: valideren van het competentieframework


\section{analyse van competenties en requirements }%
\label{sec:analyse-van-competenties-en-requirements}

In de eerste fase wordt onderzocht welke kennis en vaardigheden relevant zijn voor startende mainframeontwikkelaars met focus op PL/I en CICS. 
Daarnaast wordt in deze fase bepaald aan welke functionele en inhoudelijke vereisten de proof of concept moet voldoen.
Deze analyse vormt de basis voor de verdere uitwerking van zowel het framework als het leerplatform.



