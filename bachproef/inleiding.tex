%%=============================================================================
%% Inleiding
%%=============================================================================

\chapter{\IfLanguageName{dutch}{Inleiding}{Introduction}}%
\label{ch:inleiding}

%De inleiding moet de lezer net genoeg informatie verschaffen om het onderwerp te begrijpen en in te zien waarom de onderzoeksvraag de moeite waard is om te onderzoeken. In de inleiding ga je literatuurverwijzingen beperken, zodat de tekst vlot leesbaar blijft. Je kan de inleiding verder onderverdelen in secties als dit de tekst verduidelijkt. Zaken die aan bod kunnen komen in de inleiding~\autocite{Pollefliet2011}:

%\begin{itemize}
%  \item context, achtergrond
%  \item afbakenen van het onderwerp
%  \item verantwoording van het onderwerp, methodologie
%  \item probleemstelling
%  \item onderzoeksdoelstelling
%  \item onderzoeksvraag
%  \item \ldots
%\end{itemize}

\section{\IfLanguageName{dutch}{Probleemstelling}{Problem Statement}}%
\label{sec:probleemstelling}
%Uit je probleemstelling moet duidelijk zijn dat je onderzoek een meerwaarde heeft voor een concrete doelgroep. De doelgroep moet goed gedefinieerd en afgelijnd zijn. Doelgroepen als ``bedrijven,'' ``KMO's'', systeembeheerders, enz.~zijn nog te vaag. Als je een lijstje kan maken van de personen/organisaties die een meerwaarde zullen vinden in deze bachelorproef (dit is eigenlijk je steekproefkader), dan is dat een indicatie dat de doelgroep goed gedefinieerd is. Dit kan een enkel bedrijf zijn of zelfs één persoon (je co-promotor/opdrachtgever).

De IT sector is een snel veranderende sector waar men steeds opzoek gaat naar de nieuwste technologien om een probleem op te lossen.
Dit is toch de mindset dat het grootste deel van deze wereld heeft wanneer men het heeft over IT.
Desondaks is een van de belangerijkste onderdelen van deze sector juist heel statich en verkiest men stabiliteit boven nieuwe technologien.
Hier spreekt men over de mainframe, deze machine word al decennium lang gebruik om critieke workloads van bedrijfven met veel transacties te runnen.
Deze bedrijven zijn banken, verzekeraars, vliegtuig maatschappijen etc, zeer belangerijke bedrijven binnen onze maatschappij.
De mainframe maakt nog steeds gebruik van oude technologie zoals PL1 en CICS, waarmee hij zijn grootsete workload verzet.
Het probleem waar we de laaste 10 jaar tegenaan lopen, is de vergrijzing van het werkveld in deze sector.
Universiteiten en Hogescholen hebben hun curricula door de jaren steeds veranderd om de meest actuele onderwerpen aan studenten te kunnen bijbrengen.
Technologien zoals PL1 en CICS zijn hierdoor al meer dan 20 jaar niet meer te vinden in deze informatica opleidingen.
Door dit tekort aan leeropportuniteiten maar grote vraag voor nieuw talent bevinden studenten zich in een moeilijke positie door het tekort aan mogelijkheden tot zelfstudie.
De batchlorproef is dus gericht op nieuwe studenten en voornamelijk diegene die een internship binnen bedrijven zoals euroclear of acelormital gaan volgen want deze bedrijven maken gebruik van de PL1 technologie.

Voor deze studenten te helpen is een competentie framework nodig dat als basis kan dienen voor het ontwikkelen van leermateriaal.


\section{\IfLanguageName{dutch}{Onderzoeksvraag}{Research question}}%
\label{sec:onderzoeksvraag}
%Wees zo concreet mogelijk bij het formuleren van je onderzoeksvraag. Een onderzoeksvraag is trouwens iets waar nog niemand op dit moment een antwoord heeft (voor zover je kan nagaan). Het opzoeken van bestaande informatie (bv. ``welke tools bestaan er voor deze toepassing?'') is dus geen onderzoeksvraag. Je kan de onderzoeksvraag verder specifiëren in deelvragen. Bv.~als je onderzoek gaat over performantiemetingen, dan 

De Batchlorproef bevat één hoofdvraag met 3 deelvragen, de hoofdvraag luid :

“Hoe kan een competentieframework voor PL/I en
CICS op z/OS ontwikkeld worden dat als basis
kan dienen voor een leerplatform voor startende
mainframeontwikkelaars bij Euroclear?”


Deze vraag kan je niet beantwoorden met een
simpel antwoord, daarom stel ik volgende deel-
vragen:

• Welke competenties zijn essentieel voor een
beginnende PL/I en CICS developer binnen
een z/OS-omgeving?

Hiermee gaat opzoek gegaan worden naar wat de belangrijkste competenties zijn en kunnen we dit als basis gebruiken voor het opstellen van het framework.


• Hoe kan een competentieframework wor-
den opgebouwd dat aansluit bij de actuele
noden van het werkveld?

Naast de essentiële competenties is het bekijken van de noden van het werkveld ook uiterst belangerijkste om een compleet framework te kunnen opstellen dat actueel en meetbaar is.

• Hoe kan een proof-of-concept leerplatform
op basis van dit competentieframework meet-
baar bijdragen aan het verkleinen van de
kenniskloof en het verkorten van de tijd tot
inzetbaarheid van startende mainframeont-
wikkelaars?

Tenslotte word hier gekeken hoe zo een proof-of-concept leerplatform een concrete impact kan hebben in de actuele bedrijfs wereld.

\section{\IfLanguageName{dutch}{Onderzoeksdoelstelling}{Research objective}}%
\label{sec:onderzoeksdoelstelling}
%Wat is het beoogde resultaat van je bachelorproef? Wat zijn de criteria voor succes? Beschrijf die zo concreet mogelijk. Gaat het bv.\ om een proof-of-concept, een prototype, een verslag met aanbevelingen, een vergelijkende studie, enz.


Het beoogde resultaat van de bachelor proef
Is het afleveren van een actueel en meetbaar competentie framework.
Daarnaast wordt ook een proof-of-concept learning site ontwikkeld gebaseerd op dit competentieframework, 
waar studenten alle nodige skills kunnen leren om succesvol te zijn in het huidige werkveld rond PL1 en CICS.

De succescriteria van deze bachelorproef zijn:
\begin{itemize}
    \item Het competentieframework is \textbf{actueel en relevant}: de inhoud is afgestemd op realistische verwachtingen binnen organisaties zoals Euroclear die met PL/I en CICS werken.
    \item Het competentieframework is \textbf{meetbaar}: elke competentie is beschreven met duidelijke niveaus of indicatoren, zodat het zichtbaar is wanneer een student de competentie beheerst.
    \item Het competentieframework is \textbf{compleet en bevat alle nodige aspecten}: het framework bevat ten minste alle aspecten dat een startende pl1/CICS developer nodig heeft binnen bedrijven zoals Euroclear.
    \item Het proof-of-concept leerplatform \textbf{implementeert het competentiekader}: competenties zijn terug te vinden als leerdoelen of modules, en studenten kunnen gericht kiezen wat ze nog moeten leren.
    \item Het leerplatform bevat \textbf{concrete leerinhoud en oefeningen}: er moet een zekere aantal oefeningen zijn per hoofdstuk en een voledige leerinhoud zodat de student zich kan verdiepen en testen in de materie.
    \item Het leerplatform maakt \textbf{vooruitgang zichtbaar}: studenten kunnen op een eenvoudige manier zien welke onderdelen afgerond zijn en welke competenties nog aan gewerkt moet worden.
    \item Zowel het competentieframework als de proof-of-concept leersite zijn \textbf{bruikbaar en begrijpelijk}: een student kan ermee aan de slag zonder extra uitleg, en de link met stagevoorbereiding is duidelijk.
\end{itemize}







\section{\IfLanguageName{dutch}{Opzet van deze bachelorproef}{Structure of this bachelor thesis}}%
\label{sec:opzet-bachelorproef}

% Het is gebruikelijk aan het einde van de inleiding een overzicht te
% geven van de opbouw van de rest van de tekst. Deze sectie bevat al een aanzet
% die je kan aanvullen/aanpassen in functie van je eigen tekst.




De rest van deze bachelorproef is als volgt opgebouwd:

In Hoofdstuk~\ref{ch:stand-van-zaken} wordt een overzicht gegeven van de stand van zaken binnen het onderzoeksdomein, op basis van een literatuurstudie.

In Hoofdstuk~\ref{ch:methodologie} wordt de methodologie toegelicht en worden de gebruikte onderzoekstechnieken besproken om een antwoord te kunnen formuleren op de onderzoeksvragen.

% TODO: Vul hier aan voor je eigen hoofstukken, één of twee zinnen per hoofdstuk

In Hoofdstuk~\ref{ch:conclusie}, tenslotte, wordt de conclusie gegeven en een antwoord geformuleerd op de onderzoeksvragen. Daarbij wordt ook een aanzet gegeven voor toekomstig onderzoek binnen dit domein.