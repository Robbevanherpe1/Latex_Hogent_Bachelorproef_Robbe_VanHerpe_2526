%%=============================================================================
%% Voorwoord
%%=============================================================================

\chapter*{\IfLanguageName{dutch}{Woord vooraf}{Preface}}%
\label{ch:voorwoord}

%% TODO:
%% Het voorwoord is het enige deel van de bachelorproef waar je vanuit je
%% eigen standpunt (``ik-vorm'') mag schrijven. Je kan hier bv. motiveren
%% waarom jij het onderwerp wil bespreken.
%% Vergeet ook niet te bedanken wie je geholpen/gesteund/... heeft

Mijn bachelorproef is gericht op nieuwe geïnteresseerde studenten voor mainframe.
Ik hoop hen met deze BP een duidelijke en bruikbare basis te kunnen geven, 
zoals ik die zelf ook graag gehad zou hebben toen ik eraan begon. 
Mijn doel is om de drempel lager te maken en de eerste stappen makkelijker en minder overweldigend te laten aanvoelen.
Ik wil hiervoor zeker mr. Leendert Blondeel bedanken. Dankzij zijn uitleg, feedback en de steun tijdens het jaar kon ik veel gerichter werken.
Ook de kansen die ik kreeg binnen de mainframe wereld dankzij mr. Leendert Blondeel waren zeer waardevol en gaven me een kans om dieper in de onderwerpen zoals CICS en PL1 te gaan.
Daarnaast wil ik Meghmet Karademir bedanken voor de technische hulp en begeleiding. 
Vooral wanneer ik ergens vastzat of met vragen zit, was zijn input en uitleg heel belangerijk om mijn visie op het juiste pad te houden.
En ten slotte ook een grote dankjewel aan Euroclear om mij zo goed te ontvangen. 
Z gaven me de kans om PL1 en CICS in een echte omgeving te zien waardoor ik hier meer kan over leren, het maakte het ook direct een pak concreter.
Het heeft mij ook extra gemotiveerd om verder te groeien in dit veld.