%---------- Inleiding ---------------------------------------------------------

% TODO: Is dit voorstel gebaseerd op een paper van Research Methods die je
% vorig jaar hebt ingediend? Heb je daarbij eventueel samengewerkt met een
% andere student?
% Zo ja, haal dan de tekst hieronder uit commentaar en pas aan.

%\paragraph{Opmerking}

% Dit voorstel is gebaseerd op het onderzoeksvoorstel dat werd geschreven in het
% kader van het vak Research Methods dat ik (vorig/dit) academiejaar heb
% uitgewerkt (met medesturent VOORNAAM NAAM als mede-auteur).
% 

\section{Inleiding}%
\label{sec:inleiding}

Binnen de IT-sector wordt voortdurend gekeken naar de nieuwste technologieën, 
onderwijsinstellingen passen hun curricula aan om deze nieuwe innovaties te kunnen ondersteunen.
Door deze verandering in focus naar nieuwe technologieën is er voor mainframe-technologieën al een lange periode weinig aandacht.
Hierdoor is er een groot tekort aan interesse gekomen voor deze industrie bij studenten in de IT,
wat dan leidt tot een tekort aan nieuw talent om deze cruciale systemen operationeel te houden. 

Maar eigenlijk is het beeld dat de mainframe verouderd is en niet meer relevant zeer onterecht.
De huidige mainframe is een van de meest geavanceerde computers ter wereld en blijft een cruciaal deel van veel financiële instellingen en luchtvaartmaatschappijen,
zij draaien hun meest kritieke activiteiten nog steeds op dit platform.
Het grote probleem bevindt zich in het feit dat de gemiddelde leeftijd van professionals in dit vakgebied zeer hoog ligt,
wat zorgt voor een zeker risico dat bedrijfsspecifieke kennis verloren gaat bij bedrijven zoals Euroclear.
Wanneer een ervaren specialist met pensioen gaat zal er een groot deel van historische kennis en technische expertise verloren gaan,
enkel wanneer ze deze kennis hebben kunnen doorgeven aan een nieuw talent kan dit vermeden worden.

\subsection{Probleemstelling en onderzoeksvragen}
We kunnen hieruit de centrale probleemstelling afleiden,
er is \textcolor{red}{een groeiende kloof tussen de vraag naar nieuwe mainframeontwikkelaars en de hoeveelheid nieuw en opgeleid talent dat op de markt komt.}
Doordat onderwijsinstellingen bijna geen mainframe-specifiek aanbod hebben is de instroom van nieuw talent bijzonder laag.

De hoofdvraag van dit onderzoek is dan ook \textcolor{red}{Hoe kan een competentieframework voor PL/I en CICS op z/OS  
ontwikkeld worden dat als basis kan dienen voor een leerplatform voor startende mainframeontwikkelaars bij Euroclear?}

Deze vraag kan je niet beantwoorden met een simpel antwoord, daarom stel ik volgende deelvragen:
\textcolor{red}{
\begin{itemize}
  \item Welke competenties zijn essentieel voor een beginnende PL/I en  
        CICS developer binnen een z/OS-omgeving?  
  \item Hoe kan een competentieframework worden opgebouwd dat aansluit  
        bij de actuele noden van het werkveld?  
  \item Hoe kan een proof-of-concept leerplatform op basis van dit competentieframework meetbaar bijdragen aan het verkleinen van 
        de kenniskloof en het verkorten van de tijd tot inzetbaarheid van startende mainframeontwikkelaars? 
\end{itemize}
}

\subsection{Doelgroep}
Deze bachelorproef is gericht op studenten van HOGENT en startende mainframe ontwikkelaars. 
Het kan hen een eerste basis geven voor een toekomstige job binnen dit vakgebied.
Maar het biedt ook een zekere bijdrage aan bedrijven zoals Euroclear en HOGENT,
zij kunnen dit framework toepassen om hun interne opleidingen te versterken en te valideren.
Daarnaast kunnen organisaties de leerplatformsite gebruiken om hun nieuw talent op te leiden zodat zij beschikken over een zekere basis in PL1 en CICS op z/OS.

\subsection{Doelstelling}
De doelstelling van mijn onderzoek is het ontwikkelen van een compleet en actueel competentieframework voor PL1/CICS developers op z/OS.
Hiermee wil ik een proof-of-concept leerplatform ontwikkelen dat nieuw talent in staat stelt om zich te verdiepen in de materie,
en zich voor te bereiden op een carrière in deze sector.
Dit proof-of-concept leerplatform zou de tijd tot inzetbaarheid van startende mainframeontwikkelaars moeten verkleinen,
wat zou zorgen voor een succesvolle conclusie van mijn onderzoek.


%Waarover zal je bachelorproef gaan? Introduceer het thema en zorg dat volgende zaken zeker duidelijk aanwezig zijn:

%\begin{itemize}
%  \item kaderen thema
%  \item de doelgroep
%  \item de probleemstelling en (centrale) onderzoeksvraag
%  \item de onderzoeksdoelstelling
%\end{itemize}


%Denk er aan: een typische bachelorproef is \textit{toegepast onderzoek}, wat betekent dat je start vanuit een concrete probleemsituatie in bedrijfscontext, een \textbf{casus}. Het is belangrijk om je onderwerp goed af te bakenen: je gaat voor die \textit{ene specifieke probleemsituatie} op zoek naar een goede oplossing, op basis van de huidige kennis in het vakgebied.

%De doelgroep moet ook concreet en duidelijk zijn, dus geen algemene of vaag gedefinieerde groepen zoals \emph{bedrijven}, \emph{developers}, \emph{Vlamingen}, enz. Je richt je in elk geval op it-professionals, een bachelorproef is geen populariserende tekst. Eén specifiek bedrijf (die te maken hebben met een concrete probleemsituatie) is dus beter dan \emph{bedrijven} in het algemeen.

%Formuleer duidelijk de onderzoeksvraag! De begeleiders lezen nog steeds te veel voorstellen waarin we geen onderzoeksvraag terugvinden.

%Schrijf ook iets over de doelstelling. Wat zie je als het concrete eindresultaat van je onderzoek, naast de uitgeschreven scriptie? Is het een proof-of-concept, een rapport met aanbevelingen, \ldots Met welk eindresultaat kan je je bachelorproef als een succes beschouwen?

%---------- Stand van zaken ---------------------------------------------------

\section{Literatuurstudie}%
\label{sec:literatuurstudie}



De huidige stand van zaken is dat de mainframe-industrie een groot skillsprobleem tegemoetgaat en dit nu al ondervindt.

Het probleem van de vergrijzing binnen de mainframe industrie is al een tijdje zichtbaar en dit wordt weerspiegeld in enkele academische papers.
In de paper van Sharma \autocite{Sharma2011TeachOrNotTeach} en Ngo-Ye \autocite{NgoYe2018TeachingMainframeSkills} wordt dan ook het tekort aan deze mainframe skills getoond.
Dit tekort wordt deels veroorzaakt door de kloof tussen de academische instellingen en de bedrijfswereld.
Maar een heel beperkt aantal universiteiten hebben een mainframe opleiding die studenten kunnen voorbereiden om een job in de mainframe industrie uit te gaan voeren,
zelfs met de hoge vraag van financiële bedrijven voor deze profielen.

Daarnaast wordt in het onderzoek van Phillips \autocite{Phillips2013Mainframe} ook gekeken naar een andere oorzaak van deze vergrijzing.
Hier werd ook gezien dat de vergrijzing hand in hand gaat met een tekort aan motivatie voor studenten om mainframe-technologieën te leren.
Studenten gaan sneller kiezen voor modernere technologieën en hierdoor is het aantal steeds erg beperkt.

Programma’s zoals Master the Mainframe (momenteel IBM Z Xplore) van IBM zijn van levensbelang om nieuw talent aan te zetten om zich te verdiepen in deze technologie door de verschillende certificaten die je er kan verkrijgen.
Door het gebruik van certificaten gaan nieuwe studenten sneller de stap zetten om het zelf eens te proberen.

Door dit alles is er een grote nood aan meer van deze leermogelijkheden, maar deze moeten dan wel van een sterke basis worden opgebouwd.
Voor deze basis is een competentieframework zeer belangrijk en daar heeft IBM als grote belanghebbende van de mainframe dan ook enkele voorbeelden uitgebracht, waaronder IBM Z Systems Administrator Level 1 en Level 2 \autocite{ibm_z_sysadmin_level1,ibm_z_sysadmin_level2,ibm_mainframe_apprentice_2023}.
Deze competentieframeworks zijn vooral gericht op de systeemkant van de mainframe en bevatten daardoor beperkte meerwaarde bij het ontwikkelen van leermateriaal voor startende Z/OS developers. 
Maar deze frameworks vormen wel een duidelijke richting en opmaak om het ontwikkelen van een competentieframework voor PL1/CICS developers te begeleiden.

Daarnaast heeft IBM nog een framework dat meer toepasselijk is het Application Developer on IBM Z competentieframework \autocite{ibm_app_dev_z_2023}.
Dit framework bevat alle competenties die een developer nodig heeft maar is een te algemeen framework.
Het geeft in grote lijnen de belangrijkste onderdelen die een developer nodig heeft.
Maar dit framework is sterk gebaseerd op COBOL en binnen bedrijven zoals Euroclear is een PL1/CICS framework veel interessanter.
Een PL1/CICS developer op z/OS framework is dus nog steeds een gebied waar zo een overzicht mist.



%Hier beschrijf je de \emph{state-of-the-art} rondom je gekozen onderzoeksdomein, d.w.z.\ een inleidende, doorlopende tekst over het onderzoeksdomein van je bachelorproef. Je steunt daarbij heel sterk op de professionele \emph{vakliteratuur}, en niet zozeer op populariserende teksten voor een breed publiek. Wat is de huidige stand van zaken in dit domein, en wat zijn nog eventuele open vragen (die misschien de aanleiding waren tot je onderzoeksvraag!)?

%Je mag de titel van deze sectie ook aanpassen (literatuurstudie, stand van zaken, enz.). Zijn er al gelijkaardige onderzoeken gevoerd? Wat concluderen ze? Wat is het verschil met jouw onderzoek?


%Verwijs bij elke introductie van een term of bewering over het domein naar de vakliteratuur, bijvoorbeeld~\autocite{Hykes2013}! Denk zeker goed na welke werken je refereert en waarom.

%Draag zorg voor correcte literatuurverwijzingen! Een bronvermelding hoort thuis \emph{binnen} de zin waar je je op die bron baseert, dus niet er buiten! Maak meteen een verwijzing als je gebruik maakt van een bron. Doe dit dus \emph{niet} aan het einde van een lange paragraaf. Baseer nooit teveel aansluitende tekst op eenzelfde bron.

%Als je informatie over bronnen verzamelt in JabRef, zorg er dan voor dat alle nodige info aanwezig is om de bron terug te vinden (zoals uitvoerig besproken in de lessen Research Methods).

% Voor literatuurverwijzingen zijn er twee belangrijke commando's:
% \autocite{KEY} => (Auteur, jaartal) Gebruik dit als de naam van de auteur
%   geen onderdeel is van de zin.
% \textcite{KEY} => Auteur (jaartal)  Gebruik dit als de auteursnaam wel een
%   functie heeft in de zin (bv. ``Uit onderzoek door Doll & Hill (1954) bleek
%   ...'')

%Je mag deze sectie nog verder onderverdelen in subsecties als dit de structuur van de tekst kan verduidelijken.

%---------- Methodologie ------------------------------------------------------
\section{Methodologie}%
\label{sec:methodologie}

Voor het succesvol uitvoeren van mijn onderzoek zijn er enkele deliverables die moeten worden voldaan binnen een bepaalde tijdspanne, deze omvatten:

\begin{itemize}
  \item 3 feb 2026 - Indienen draft BP met inleiding literatuurstudie en methodologie.
  \item 4 mei 2026 - Indienen finale draft bachelorproef.
  \item 29 mei 2026 - Indienen bachelorproef.
\end{itemize}

Daarom ga ik te werk in volgende fases om deze deliverables op het juiste moment te kunnen leveren.

Overzicht van fases in Gantt Chart:

\includegraphics[width=0.45\textwidth]{Gantt Chart.png}

\subsection{Literatuurstudie en interviews}

In de eerste fase ga ik een literatuurstudie uitvoeren aan de hand van papers en artikels over PL1 en CICS zoals "CICS Transaction Processing on zOS: Core Concepts and Workflow"\autocite{YalamanchiliIJLRP} en "Analyzing PL/1 Legacy Ecosystems: An Experience Report"\autocite{pl1legacy2021}.
Daarnaast ga ik een analyse maken van voorgaande competentieframeworks en studies om tot een concrete lijst van requirements te bekomen.
Deze voorlopige lijst wordt vervolgens verfijnd aan de hand van semi-gestructureerde interviews met twee tot drie professionals uit Euroclear en eventueel één of twee externe mainframe-experts.
Ik kies hier voor semi-gestructureerde interviews, deze laten toe dat er gerichte vragen worden gesteld, maar geven ook ruimte voor bijkomende inzichten over nodige competenties.
De verzamelde informatie wordt vervolgens geordend volgens het MoSCoW-principe om te komen tot een gestructureerde lijst met de belangrijkste requirements (Must haves en Should haves).
Tot slot wordt deze lijst met requirements terug voorgelegd aan deze experts om gerichte feedback te krijgen en de requirements te valideren.
Wanneer uit deze validatie blijkt dat er nog grote gaten zitten in mijn opgestelde lijst van requirements, dan zal deze moeten worden herzien.
Voor deze fase verwacht ik dan ook een duur van ongeveer vier weken met als deadline 2 maart 2026.

\subsection{Opstellen framework en PoC}
De volgende fase van mijn onderzoek zal het opstellen van een compleet competentieframework bevatten dat als basis kan liggen bij het uitwerken van nieuw leermateriaal.
Dit is zeer relevant voor organisaties en beginnende ontwikkelaars want dit geeft hen een duidelijke leidraad om PL1 en CICS stap voor stap aan te leren, in plaats van te moeten werken met losse kennis en ongestructureerde documentatie.
Het opstellen van dit framework ga ik doen aan de hand van de lijst met requirements, die alle noodzakelijke elementen bevat.
Ik ga ook gebruik maken van de voorgaande competentieframeworks zoals het developer framework van IBM \autocite{ibm_app_dev_z_2023} om een correcte opbouw en stijl te hebben in mijn competentieframework voor PL1/CICS developers.

Dit competentieframework zou dan als een sterke basis moeten dienen voor het uitvoeren van mijn volgende fases en organisaties in staat stellen om hun eigen leermateriaal te ontwikkelen specifiek voor PL1 en CICS.
Concreet moet het framework het mogelijk maken om leerdoelen af te leiden en die om te zetten naar lessen of oefeningen.

Naast het uitwerken van dit framework ga ik ook aan de slag om een proof-of-concept leerplatform te maken dat gebaseerd is op dit competentieframework.
Mijn doel met dit leerplatform is om na te gaan of er duidelijke leerdoelen en lessen kunnen worden afgeleid van het framework om hiermee de site van lesmateriaal te voorzien.
Daarnaast wil ik studenten of beginnende ontwikkelaars de kans geven om zich te ontwikkelen in deze technologieën.
Het leerplatform zal slechts een beperkt aantal features bevatten en vooral dienen als testomgeving voor het competentieframework.

Indien uit dit praktische voorbeeld blijkt dat het afleiden van leerdoelen en het opbouwen van lessen succesvol is,
en als het leerplatform volgens experts voldoet aan de vooropgestelde eisen, dan beschouw ik mijn framework als een succes.

Voor het uitwerken van deze fase en het te bekomen van een competentieframework en proof-of-concept ga ik hoofdzakelijk gebruik maken van volgende technologieën.

\begin{itemize}
  \item Excel voor het opstellen van het competentieframework.
  \item Github voor de versiecontrole van mijn poc site.
  \item HTML,CSS,JS voor de structuur, interactiviteit en opmaak van mijn poc site.
  \item Github Pages voor het hosten van mijn site.
\end{itemize}

Deze tweede fase zou klaar moeten zijn na zes weken met als deadline 4 mei 2026.

\subsection{Rapporteren paper}
Tenslotte in de laatste fase verwerk ik al deze resultaten in een paper als finale versie van mijn bachelorproef.
Deze fase loopt eigenlijk gedurende het hele bachelorproefproces maar wordt gefinaliseerd in de laatste twee weken na het indienen van de draft.
De deadline voor deze laatste fase is 29 mei 2026.


%Hier beschrijf je hoe je van plan bent het onderzoek te voeren. Welke onderzoekstechniek ga je toepassen om elk van je onderzoeksvragen te beantwoorden? Gebruik je hiervoor literatuurstudie, interviews met belanghebbenden (bv.~voor requirements-analyse), experimenten, simulaties, vergelijkende studie, risico-analyse, PoC, \ldots?

%Valt je onderwerp onder één van de typische soorten bachelorproeven die besproken zijn in de lessen Research Methods (bv.\ vergelijkende studie of risico-analyse)? Zorg er dan ook voor dat we duidelijk de verschillende stappen terug vinden die we verwachten in dit soort onderzoek!

%Vermijd onderzoekstechnieken die geen objectieve, meetbare resultaten kunnen opleveren. Enquêtes, bijvoorbeeld, zijn voor een bachelorproef informatica meestal \textbf{niet geschikt}. De antwoorden zijn eerder meningen dan feiten en in de praktijk blijkt het ook bijzonder moeilijk om voldoende respondenten te vinden. Studenten die een enquête willen voeren, hebben meestal ook geen goede definitie van de populatie, waardoor ook niet kan aangetoond worden dat eventuele resultaten representatief zijn.

%Uit dit onderdeel moet duidelijk naar voor komen dat je bachelorproef ook technisch voldoen\-de diepgang zal bevatten. Het zou niet kloppen als een bachelorproef informatica ook door bv.\ een student marketing zou kunnen uitgevoerd worden.

%Je beschrijft ook al welke tools (hardware, software, diensten, \ldots) je denkt hiervoor te gebruiken of te ontwikkelen.

%Probeer ook een tijdschatting te maken. Hoe lang zal je met elke fase van je onderzoek bezig zijn en wat zijn de concrete \emph{deliverables} in elke fase?

%---------- Verwachte resultaten ----------------------------------------------
\section{Verwacht resultaat, conclusie}%
\label{sec:verwachte_resultaten}

Het verwachte resultaat van mijn bachelorproef is het opstellen van een actueel en compleet competentieframework voor PL1 en CICS developer op z/Os level 1.
Met dit framework gaat er een duidelijk overzicht zijn van welke competenties nodig zijn om een succesvolle z/OS developer te zijn binnen cruciale bedrijven.
Dit geeft andere onderzoekers en bedrijven de kans om dit framework als basis te gebruiken om hun leeromgeving te vormen voor toekomstige developers.

Aansluitend verwacht ik ook een proof-of-concept site gebaseerd op dit opgesteld competentieframework voor PL1 en CICS developers.
Aan de hand van deze learningsite zouden toekomstige z/Os developers een basis kunnen aanleggen in hun theoretische skills met PL1 en CICS maar ook praktische aan de hand van hands on labs/examens.
Deze combinatie zorgt ervoor dat ze een waardevolle toevoeging zijn binnen bedrijven die gebruikmaken van een mainframe.



%Hier beschrijf je welke resultaten je verwacht. Als je metingen en simulaties uitvoert, kan je hier al mock-ups maken van de grafieken samen met de verwachte conclusies. Benoem zeker al je assen en de onderdelen van de grafiek die je gaat gebruiken. Dit zorgt ervoor dat je concreet weet welk soort data je moet verzamelen en hoe je die moet meten.

%Wat heeft de doelgroep van je onderzoek aan het resultaat? Op welke manier zorgt jouw bachelorproef voor een meerwaarde?

%Hier beschrijf je wat je verwacht uit je onderzoek, met de motivatie waarom. Het is \textbf{niet} erg indien uit je onderzoek andere resultaten en conclusies vloeien dan dat je hier beschrijft: het is dan juist interessant om te onderzoeken waarom jouw hypothesen niet overeenkomen met de resultaten.

