%==============================================================================
% Sjabloon onderzoeksvoorstel bachproef
%==============================================================================
% Gebaseerd op document class `hogent-article'
% zie <https://github.com/HoGentTIN/latex-hogent-article>

% Voor een voorstel in het Engels: voeg de documentclass-optie [english] toe.
% Let op: kan enkel na toestemming van de bachelorproefcoördinator!
\documentclass{hogent-article}

% Invoegen bibliografiebestand
\addbibresource{voorstel.bib}

% Informatie over de opleiding, het vak en soort opdracht
\studyprogramme{Professionele bachelor toegepaste informatica}
\course{Bachelorproef}
\assignmenttype{Onderzoeksvoorstel}
% Voor een voorstel in het Engels, haal de volgende 3 regels uit commentaar
% \studyprogramme{Bachelor of applied information technology}
% \course{Bachelor thesis}
% \assignmenttype{Research proposal}

\academicyear{2025-2026} % TODO: pas het academiejaar aan

% TODO: Werktitel
\title{Competency framework PL/I CICS developer op z/OS level 1 definiëren en uitwerken met PoC binnen onderwijscontext.}

% TODO: Studentnaam en emailadres invullen
\author{Robbe Van Herpe}
\email{robbe.vanherpe@student.hogent.be}

% TODO: Medestudent
% Gaat het om een bachelorproef in samenwerking met een student in een andere
% opleiding? Geef dan de naam en emailadres hier
% \author{Yasmine Alaoui (naam opleiding)}
% \email{yasmine.alaoui@student.hogent.be}

% TODO: Geef de co-promotor op
\supervisor[Co-promotor]{M. Karademir (Euroclear, \href{mailto:mehmet.karademir@euroclear.com}{mehmet.karademir@euroclear.com})}

% Binnen welke specialisatierichting uit 3TI situeert dit onderzoek zich?
% Kies uit deze lijst:
%
% - Mobile \& Enterprise development
% - AI \& Data Engineering
% - Functional \& Business Analysis
% - System \& Network Administrator
% - Mainframe Expert
% - Als het onderzoek niet past binnen een van deze domeinen specifieer je deze=
%   zelf
%
\specialisation{Mainframe Expert}
\keywords{PL1, CICS, Competency Framework}

\begin{document}

\begin{abstract}
Onderwijsinstellingen binnen de IT-sector richten zich steeds meer op de nieuwste technologieën.
Hierdoor is er steeds minder plaats voor opleidingen rond mainframe-technologieën. 
Dit fenomeen speelt al jaren binnen de mainframesector en heeft geleid tot een sterke vergrijzing onder het actieve werkveld. 
Voor bedrijven zoals Euroclear die op deze technologieën steunen, vormt dit een groot probleem, omdat waardevolle historische en bedrijfsspecifieke kennis dreigt verloren te gaan.
Een van de belangrijkste dingen om dit te voorkomen is het opleiden van nieuw mainframe talent.
De hoofdonderzoeksvraag voor mijn onderzoek gaat dan als volgt: 
Hoe kan een competentieframework voor PL/I en CICS op z/OS ontwikkeld worden dat als basis kan dienen voor een leerplatform voor startende mainframeontwikkelaars bij Euroclear?

Het onderzoek bestaat uit een literatuurstudie en interviews met experts om de nodige requirements voor een competentieframework te bepalen.
Op basis van deze requirements wordt een framework opgesteld en ga ik aan het werk om een praktische learningsite te ontwikkelen gebaseerd op dit competentieframework.
Deze site moet het competentieframework valideren en nieuw talent de kans geven om praktijkervaring op te doen.
Het verwachte resultaat van mijn onderzoek is een bijdrage aan de groei van nieuw talent met kennis van mainframe-technologieën, 
door middel van theoretische kennis en praktische oefeningen op de learningsite om hen de kans te geven zich te verdiepen in deze technologie.
Daarbovenop gaat het competentieframework een hulpmiddel zijn voor bedrijven zoals Euroclear en HOGENT om hun opleidingsmateriaal te versterken op basis van een actueel en volledig competentieframework.


 %  Hier schrijf je de samenvatting van je voorstel, als een doorlopende tekst van één paragraaf.
 %  Let op: dit is geen inleiding, maar een samenvattende tekst van heel je voorstel met inleiding (voorstelling, kaderen thema),
 %  probleemstelling en centrale onderzoeksvraag, onderzoeksdoelstelling (wat zie je als het concrete resultaat van je bachelorproef?),
 %  voorgestelde methodologie, verwachte resultaten en meerwaarde van dit onderzoek (wat heeft de doelgroep aan het resultaat?).
\end{abstract}

\tableofcontents

% De hoofdtekst van het voorstel zit in een apart bestand, zodat het makkelijk
% kan opgenomen worden in de bijlagen van de bachelorproef zelf.
%---------- Inleiding ---------------------------------------------------------

% TODO: Is dit voorstel gebaseerd op een paper van Research Methods die je
% vorig jaar hebt ingediend? Heb je daarbij eventueel samengewerkt met een
% andere student?
% Zo ja, haal dan de tekst hieronder uit commentaar en pas aan.

%\paragraph{Opmerking}

% Dit voorstel is gebaseerd op het onderzoeksvoorstel dat werd geschreven in het
% kader van het vak Research Methods dat ik (vorig/dit) academiejaar heb
% uitgewerkt (met medesturent VOORNAAM NAAM als mede-auteur).
% 

\section{Inleiding}%
\label{sec:inleiding}

Binnen de IT sector wordt voortdurend gekeken naar de nieuwste technologieën, 
onderwijsinstellingen passen hun curriculums aan om deze nieuwe technologieën te kunnen ondersteunen.
Door deze verandering in focus naar nieuwe technologieën is er voor mainframe technologieën al een lange periode weinig aandacht.
Hierdoor is er een groot tekort aan intresse gekomen voor deze industrie bij studenten in de IT, want dan leid tot een tekort aan nieuw talent om deze cruciale machines te blijven laten werken. 


Maar dat is eigenlijk niet terecht het beeld dat deze technologie verouderd en niet meer relevant is.
De mainframe blijf een cruciaal deel van veel financiële instellingen en luchtvaartmaatshapijene, zij draaien hun meest critieke activitetien nog steed op dit platform.
Het grote probleem vind zich dat de gemiddelde leeftijd van de profesionals in dit vakgebied zeer hoog ligt,
dit zorgt voor een zeker risico dat bedrijfspecifieke kennis verloren gaat gaan bij bedrijven zoals Euroclear.
Wanneer een ervaren specialist met pensioen gaat zal er een groot deel van historishe kennis en technisch expertise verloren gaan.
Enkel wannneer ze deze kennis hebben kunnen doorgeven aan een nieuw talent kan dit vermeden worden.

\subsection{Probleemstelling en onderzoeksvragen}
We kunnen dan ook de centrale probleem stelling hieruit afleiden,
er is \textcolor{red}{een groeiende kloof tussen de vraag naar nieuw mainframe ontiwkkelaars en de hoeveelheid nieuw en opgeleid talent dat op de markt komt.}
Doordat onderwijsinstellingen bijna geen mainframe specifiek aanbod hebben is de instroom van nieuw talent bizonder laag.

De hoofdvraag van dit onderzoek is dan ook \textcolor{red}{Hoe kan een competentieframework voor PL/I en CICS op z/OS  
ontwikkeld worden dat als basis kan dienen voor een leerplatform voor startende mainframeontwikkelaars bij Euroclear?”}

Deze vraag an je niet beantwoorden met een simpel antwoord, daarom stel ik volgende deelvragen:
\textcolor{red}{
\begin{itemize}
  \item Welke competenties zijn essentieel voor een beginnende PL/I en  
        CICS developer binnen een z/OS-omgeving?  
  \item Hoe kan een competentieframework worden opgebouwd dat aansluit  
        bij de actuele noden van het werkveld?  
  \item Hoe kan een proof-of-concept leerplatform op basis van dit  
        framework bijdragen aan het verkleinen van de skill gap?  
\end{itemize}
}

\subsection{Doelgroep}
Deze bachelorproef is dan ook gericht op studenten van HOGENT en startende mainframe ontwikkelaars. 
Het kan hen een eerste basis geven voor een toekomstige job binne dit vakgebied.
Maar het geeft ook een zekere bijdrage aan bedrijven zoals Euroclear en Hogent,
zij kunnen dit framwork toepassen om hun interne opleidingen te versterken en te valideren.
Daarnaast kunnen oragnisaties de leerplatform site gebruiken om hun nieuw talent op te leiden zodat zij beschikken over een zekere basis in dit gebied.

\subsection{Doelstelling}
De doelstelling van mijn onderzoek is het ontwikkelen van een compleet en actueel competentieframework voor PL1/CICS developers op Z/oS.
Hiermee wil ik een proof of concept leerplatform ontwikkelen dat nieuw talent in staat stelt om zich te verdiepen in de materie,
en zich voor te bereiden op een cariere in deze sector.
Dit zou zorgen voor een succesvolle conclusie van mijn onderzoek.


%Waarover zal je bachelorproef gaan? Introduceer het thema en zorg dat volgende zaken zeker duidelijk aanwezig zijn:

%\begin{itemize}
%  \item kaderen thema
%  \item de doelgroep
%  \item de probleemstelling en (centrale) onderzoeksvraag
%  \item de onderzoeksdoelstelling
%\end{itemize}


%Denk er aan: een typische bachelorproef is \textit{toegepast onderzoek}, wat betekent dat je start vanuit een concrete probleemsituatie in bedrijfscontext, een \textbf{casus}. Het is belangrijk om je onderwerp goed af te bakenen: je gaat voor die \textit{ene specifieke probleemsituatie} op zoek naar een goede oplossing, op basis van de huidige kennis in het vakgebied.

%De doelgroep moet ook concreet en duidelijk zijn, dus geen algemene of vaag gedefinieerde groepen zoals \emph{bedrijven}, \emph{developers}, \emph{Vlamingen}, enz. Je richt je in elk geval op it-professionals, een bachelorproef is geen populariserende tekst. Eén specifiek bedrijf (die te maken hebben met een concrete probleemsituatie) is dus beter dan \emph{bedrijven} in het algemeen.

%Formuleer duidelijk de onderzoeksvraag! De begeleiders lezen nog steeds te veel voorstellen waarin we geen onderzoeksvraag terugvinden.

%Schrijf ook iets over de doelstelling. Wat zie je als het concrete eindresultaat van je onderzoek, naast de uitgeschreven scriptie? Is het een proof-of-concept, een rapport met aanbevelingen, \ldots Met welk eindresultaat kan je je bachelorproef als een succes beschouwen?

%---------- Stand van zaken ---------------------------------------------------

\section{Literatuurstudie}%
\label{sec:literatuurstudie}



De huidige stand van zaken is dat de mainframe-industrie een groot skillsprobleem tegemoetgaat en dit nu al ondervindt.

Het probleem van de vergrijzing binne de mainframe industrie is al een tijdje zichtbaar en dit word weerspiegeld in enkele academishe papers.
In de paper van \autocite{Sharma2011TeachOrNotTeach} en \autocite{NgoYe2018TeachingMainframeSkills} word dan ook het tekort aan deze mainframe skills getoont.
Dit tekort word deel veroorzaakt door de kloof van de academishe instellingen en de bedrijfswereld.
Maar een heel beperkt aantal universiteiten hebben een mainframe opleiding die studenten kunnen voorbereiden om een job in de mainframe industrie uit te gaan voeren, zelfs met de hoge vraag van financiele bedrijven voor deze profielen.

Daarnaast word in het onderzoek van \autocite{Phillips2013Mainframe} ook gekeken naar een andere oorzaak van deze vergrijzing.
Hier werd dan ook gezien dat de vergrijzing ook hand in hand loop met een tekort aan motievatie voor studenten om mainframe technologien te leren.
Studenten gaan sneller keizen voor modernere technologien en hierdoor is het aantal steeds erg beperkt.

Programmas zoals Master the Mainframe (momenteel IBM Z Xplore) van IBM zijn dan ook van levenbelang om nieuw talent aan te zetten om zich te verdiepen in deze technologie door de verschillende certificaten die je er kan verkrijgen.
Door het gebruik van certificaten gaan nieuwe studenten sneller de stap zetten om het zelf eens te proberen.

Door dit alles is er dan ook een grote nood aan meer van deze leermogelijkheden, maar deze moeten dan wel van een sterke basis worden opgebouwd.
Voor deze basis is een competetie framework zeer belangerijk en daar heeft IBM als grote belanghebber van de mainframe dan ook enkele voorbeelden uitgebracht,waaronder IBM Z Systems Administrator Level 1 en Level 2 \autocite{ibm_z_sysadmin_level1,ibm_z_sysadmin_level2,ibm_mainframe_apprentice_2023}.
Deze competentieframeworks zijn vooral gericht op de systeem kant van de mainframe en bevatten daardoor beperkte meerwaarde bij het ontwikkelen van leermateriaal voor startende Z/OS developers. 
Maar deze frameworks vormen wel een duidelijke richting en opmaak om het ontwikkelen van een competentieframework voor PL1/CICS developers te begeleiden.

Daarnaast heeft IBM nog een framework dat meer toepasselijk is het Application Developer on IBM Z competentie framework \autocite{ibm_app_dev_z_2023}, dit framework bevat alle competenties die een developer nodig heeft maar is een te algemeen framework.
Het geef in grote lijnen de belangerijkste onderdelen die een developer nodig heeft.
Maar dit framework is sterk gebaseerd op cobol en binne bedrijven zoals Euroclear is een PL1/CICS framework veel interresanter.
Een PL1/CICS developer on Z/OS framework is dus nog stees een gebied waar zo een overzicht mist.



%Hier beschrijf je de \emph{state-of-the-art} rondom je gekozen onderzoeksdomein, d.w.z.\ een inleidende, doorlopende tekst over het onderzoeksdomein van je bachelorproef. Je steunt daarbij heel sterk op de professionele \emph{vakliteratuur}, en niet zozeer op populariserende teksten voor een breed publiek. Wat is de huidige stand van zaken in dit domein, en wat zijn nog eventuele open vragen (die misschien de aanleiding waren tot je onderzoeksvraag!)?

%Je mag de titel van deze sectie ook aanpassen (literatuurstudie, stand van zaken, enz.). Zijn er al gelijkaardige onderzoeken gevoerd? Wat concluderen ze? Wat is het verschil met jouw onderzoek?


%Verwijs bij elke introductie van een term of bewering over het domein naar de vakliteratuur, bijvoorbeeld~\autocite{Hykes2013}! Denk zeker goed na welke werken je refereert en waarom.

%Draag zorg voor correcte literatuurverwijzingen! Een bronvermelding hoort thuis \emph{binnen} de zin waar je je op die bron baseert, dus niet er buiten! Maak meteen een verwijzing als je gebruik maakt van een bron. Doe dit dus \emph{niet} aan het einde van een lange paragraaf. Baseer nooit teveel aansluitende tekst op eenzelfde bron.

%Als je informatie over bronnen verzamelt in JabRef, zorg er dan voor dat alle nodige info aanwezig is om de bron terug te vinden (zoals uitvoerig besproken in de lessen Research Methods).

% Voor literatuurverwijzingen zijn er twee belangrijke commando's:
% \autocite{KEY} => (Auteur, jaartal) Gebruik dit als de naam van de auteur
%   geen onderdeel is van de zin.
% \textcite{KEY} => Auteur (jaartal)  Gebruik dit als de auteursnaam wel een
%   functie heeft in de zin (bv. ``Uit onderzoek door Doll & Hill (1954) bleek
%   ...'')

%Je mag deze sectie nog verder onderverdelen in subsecties als dit de structuur van de tekst kan verduidelijken.

%---------- Methodologie ------------------------------------------------------
\section{Methodologie}%
\label{sec:methodologie}

Voor het succesvol uitvoeren van mijn onderzoek zijn er enkele deliverables die moeten voldaan worden binne een bepaalde tijdspanne, deze omvatten:

\begin{itemize}
  \item 3 feb 2026 - Indienen draft BP met inleiding literatuurstudie en methodologie.
  \item 4 mei 2026 - Indienen finale draft bachelorproef.
  \item 4 mei 2026 - Indienen bachelorproef.
\end{itemize}

Daarom ga ik tewerk in vogende fases om deze deliverabels op het juiste moment te kunnen leveren.

Overzicht van fases in Gantt Chart:

\includegraphics[width=0.5\textwidth]{Gantt Chart.png}

\subsection{Literatuurstudie en intervieuws}
In de eerste fase ga ik een literatuurstudie doen aan de hand van papers en artikels over PL1 en CICS zoals "CICS Transaction Processing on zOS: Core Concepts and Workflow"\autocite{YalamanchiliIJLRP} en "Analyzing PL/1 Legacy Ecosystems: An Experience Report"\autocite{pl1legacy2021}.
Daarnaast ga ik een analyse maken van voorgaande competentieframework en studies om tot een concrete lijst van requirements te bekomen.
Met deze lijst als leidraad ga ik enkele intervieuws houden met profesionals binnen Euroclear en andere relevante bedrijven om mijn voorgaande requirements te verfijen.
Wanneer ik merk dat er nog grote gaten zitten in mijn opgestelde lijst van requirements dan zal ik deze moeten hervormen en de nodige stukken aanvullen.
Voor deze fase verwacht ik dan ook een duur van ongeveer 4 weken met als deadline 2 maart 2026.

\subsection{Opstellen framework en PoC}
De volgende fase van mijn onderzoek zal het opstellen van een compleet competentieframework bevatten dat als basis kan liggen bij het uitwerken van nieuw leermateriaal.
Het opstellen van dit framework ga ik doen aan de hand van de lijst met requirements die alle noodzakelijke dingen zou moet bevatten.
Ik ga ook gebruik maken van de voorgaande competentie frameworks zoals het developer framework van IBM \autocite{ibm_app_dev_z_2023} om een correct opbouw en stijl te hebben in mijn competentieframework voor PL1/CICS developers.
Dit competentieframework zou dan als een sterke basis moeten dienen voor het uitvoeren van mijn volgende fases en organisaties in staat stellen om hun eigen leermateriaal te ontwikkelen specifiek voor PL1 en CICS.
Naast het uitwerken van dit framework ga ik ook aan de slag om proof of concept learnsite te maken die zich baseerd op het competieframework voor PL1 en CICS.
Mijn doel met deze learningsite is het kunne valideren van mijn competentieframework voor een praktishe aplicatie er van en studenten of beginnende ontwikkelaars de kans te geven om zich te ontwikkelen in deze technologien.
De volwaardigheid en correctheid van het competentieframework zijn dus van hoogst belang om een succesvolle poc te kunnen maken.
Indien uit het praktishe voorbeeld blijkt dat er iets niet haalbaar is in het framework kan ik dit dan ook nog aanpassen.
Voor het uitwerken van deze fase en het te bekomen van een competentieframework en proof of concept ga ik hoofdzakelijk gebruik maken van volgende technologieën.

\begin{itemize}
  \item Exel voor het opstellen van de Competency framework.
  \item Github voor de versie controle van mijn poc site.
  \item HTMl,CSS,JS voor de stuctuur, interactieviteit en opmaak van mijn poc site.
  \item Github Pages voor het hosten van mijn site.
\end{itemize}

Deze tweede fase zou klaar moeten zijn na vier weken met als deadline 4 mei 2026.

\subsection{Rapporteren Paper}
Dan tenslotte in de laaste fase verwerk ik al deze resultaten in een paper als finale versie van mijn bachelorproef.
Deze fase loopt eigenlijk gedurende het hele bachelorproef process maar word gefinalizeerd in de laatse 2 weken na het uploaden van de draft.
De deadline voor deze laaste fase is dan ook 29 mei 2026.


%Hier beschrijf je hoe je van plan bent het onderzoek te voeren. Welke onderzoekstechniek ga je toepassen om elk van je onderzoeksvragen te beantwoorden? Gebruik je hiervoor literatuurstudie, interviews met belanghebbenden (bv.~voor requirements-analyse), experimenten, simulaties, vergelijkende studie, risico-analyse, PoC, \ldots?

%Valt je onderwerp onder één van de typische soorten bachelorproeven die besproken zijn in de lessen Research Methods (bv.\ vergelijkende studie of risico-analyse)? Zorg er dan ook voor dat we duidelijk de verschillende stappen terug vinden die we verwachten in dit soort onderzoek!

%Vermijd onderzoekstechnieken die geen objectieve, meetbare resultaten kunnen opleveren. Enquêtes, bijvoorbeeld, zijn voor een bachelorproef informatica meestal \textbf{niet geschikt}. De antwoorden zijn eerder meningen dan feiten en in de praktijk blijkt het ook bijzonder moeilijk om voldoende respondenten te vinden. Studenten die een enquête willen voeren, hebben meestal ook geen goede definitie van de populatie, waardoor ook niet kan aangetoond worden dat eventuele resultaten representatief zijn.

%Uit dit onderdeel moet duidelijk naar voor komen dat je bachelorproef ook technisch voldoen\-de diepgang zal bevatten. Het zou niet kloppen als een bachelorproef informatica ook door bv.\ een student marketing zou kunnen uitgevoerd worden.

%Je beschrijft ook al welke tools (hardware, software, diensten, \ldots) je denkt hiervoor te gebruiken of te ontwikkelen.

%Probeer ook een tijdschatting te maken. Hoe lang zal je met elke fase van je onderzoek bezig zijn en wat zijn de concrete \emph{deliverables} in elke fase?

%---------- Verwachte resultaten ----------------------------------------------
\section{Verwacht resultaat, conclusie}%
\label{sec:verwachte_resultaten}

Het verwachte resultaat van mijn bachelorproef is het kunnen opstellen van een actueel en compleet competentie framework voor PL1 en CICS developer op z/Os level 1.
Met dit framework gaat er een duidelijk overzicht zijn van welke competenties nodig zijn om een succesvolle Z/os developer te zijn binnen cruciale bedrijven.
Dit geef andere onderzoekers en bedrijven de kans om dit framework als basis te gebruiken om hun leeromgeving te vormen voor toekomstige developers.

Aansluitend verwacht ik ook een proof of concept site gebaseerd op dit opgesteld competentie framework voor PL1 en CICS developers.
Aan de hand van deze learningsite zouden toekomstige z/Os developers een basis kunnen aanleggen in hun theoretieshe skills met PL1 en CICS maar ook praktishe aan de hand van hands on labs/examens.
Deze combinatie zorgt ervoor dat ze een waardevolle additie zijn binnen bedrijven die gebruik maken van een mainframe.



%Hier beschrijf je welke resultaten je verwacht. Als je metingen en simulaties uitvoert, kan je hier al mock-ups maken van de grafieken samen met de verwachte conclusies. Benoem zeker al je assen en de onderdelen van de grafiek die je gaat gebruiken. Dit zorgt ervoor dat je concreet weet welk soort data je moet verzamelen en hoe je die moet meten.

%Wat heeft de doelgroep van je onderzoek aan het resultaat? Op welke manier zorgt jouw bachelorproef voor een meerwaarde?

%Hier beschrijf je wat je verwacht uit je onderzoek, met de motivatie waarom. Het is \textbf{niet} erg indien uit je onderzoek andere resultaten en conclusies vloeien dan dat je hier beschrijft: het is dan juist interessant om te onderzoeken waarom jouw hypothesen niet overeenkomen met de resultaten.



\printbibliography[heading=bibintoc]

\end{document}